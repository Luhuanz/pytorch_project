\fancyfoot{ }
\begin{abstract}
    本文通过在MNist数据集上通过分析各种网络结构正确认识CV,具体认识与实现模型以求更加深刻的认识计算机视觉。\par
    实验的第一部分:自制了beamer主题,目的用于报告的展示。以求达到能够清晰表达的作用,为了能够提高实验的效率,本文使用Anaconda也重新安装了pytorch\_gpu版.\par
    实验的第二部分:从五部分来建模。首先通过写MLP模型,具体且清晰的实现了一般DNN建模的步骤,并且在初步微调模型得到了test集上\textbf{97\%}的成绩。接着加大网络的深度,通过建立LeNet、AlexNet丰富了报告的内容,同时得到了更高的acc,其中LeNet跑10$epoch$模型平均准确率有\textbf{97.6\%},而AlexNet更深达到了\textbf{99.26\%},也是这5个网络中最高的acc。并且在AlexNet中,分别对比了手写以及模块化的不同,同时发现优化器在选择adam的时候再第一个epoch会出现很大的误差,而且这种情况不可逆,因而得选择SGD。接着使用GoogLeNet去加大网络的深度,使用了Inception V1块,但是遗憾的是GooLeNet在跑test的时候并未得到预期的结果。在换成SGD的情况下依旧失效,未知其原因。最后,为了避免ResNet出现同样的情况。我使用目前正在使用的fastai,使用它的轮子进行\textbf{迁移学习}。分别跑了ResNet18以及ResNet34得到acc分别是99\%、98\%。模型的网络变深,test的结果反倒变差了,尽管ResNet有BN层,但是有理由相信,模型已经出现了轻微的过拟合。\par 
    实验的第三部分:在具体认识模型的同时,提供了一些提高模型准确率的数据增强手段。
    
    
\end{abstract}

\textbf{关键词:} pytorch \quad MNIST\quad GoogNet\quad ResNet\quad fastai \quad 迁移学习
\newpage
\fancyfoot[C]{\bfseries\thepage}